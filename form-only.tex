\documentclass[msc,deptreport,cs]{infthesis} % Do not change except to add your degree (see above).

%% Imports poached from frankly
%% STILL can't find what makes \figrule work
\usepackage{natbib}
\usepackage{mathpartir}
\usepackage{amsmath}
\usepackage{amsfonts}
\usepackage{thmtools,thm-restate}
\usepackage{comment}
\usepackage{flushend}
\usepackage{listings}

%% \usepackage{beramono}


%% \lstdefinestyle{mystyle}{
%%     %% backgroundcolor=\color{backcolour},
%%     %% commentstyle=\color{codegreen},
%%     %% keywordstyle=\color{magenta},
%%     %% numberstyle=\tiny\color{codegray},
%%     %% stringstyle=\color{codepurple},
%%     breakatwhitespace=false,
%%     breaklines=true,
%%     captionpos=b,
%%     keepspaces=true,
%%     %% numbers=left,
%%     %% numbersep=5pt,
%%     showspaces=false,
%%     showstringspaces=false,
%%     showtabs=false,
%%     tabsize=2,
%%     %% basicstyle=\small\ttfamily
%%     basicstyle=\ttfamily\footnotesize,
%%     %% breaklines=true
%% }

%% \lstset{style=mystyle}

\lstset{
  basicstyle=\small\ttfamily,
  breaklines=true
}

\usepackage{stmaryrd}
\usepackage{natbib}
\usepackage{xspace}
%% \usepackage[pdftex,
%%             pdfauthor={Sam Lindley, Conor McBride, and Craig McLauglin},
%%             pdftitle={Doo bee doo bee doo}]{hyperref}
%\hypersetup{colorlinks=true,citecolor=blue,linkcolor=blue}
%% \hypersetup{colorlinks=true,allcolors=black}
\usepackage[usenames,dvipsnames]{xcolor}
\usepackage{url}

% get rid of hypertext link on \citeauthor
\usepackage{etoolbox}

\usepackage{amssymb}

\usepackage{mathtools} % allows flush-left align environments and paired
                       % delimiters.
                       %


% Theorem environments
\newtheorem{theorem}{Theorem}
\newtheorem{proposition}[theorem]{Proposition}
\newtheorem{definition}[theorem]{Definition}
\newtheorem{prop}[theorem]{Proposition}
\newtheorem{lemma}[theorem]{Lemma}
\newtheorem{corollary}[theorem]{Corollary}


%% abstract for inline code
\newcommand{\code}[1]{\lstinline{#1}}

\newcommand{\hl}[1]{\colorbox{lightgray}{#1}}

\usepackage{xcolor}

\newcommand{\highlight}[1]{%
  \colorbox{red!20}{$\displaystyle#1$}}

%%%%%%%%%%%%%%%%%%%%%%%%%%%%%%%%%%%%%%%%%%%%%%%%%
%% Start of inference rules typesetting business
%%%%%%%%%%%%%%%%%%%%%%%%%%%%%%%%%%%%%%%%%%%%%%%%%

\newcommand{\counter}{{\color{blue}c_y}}
\newcommand{\justc}[1]{{\color{blue} c({#1})}}
\newcommand{\yieldc}{{\color{blue}{\textsf{yield}}}}
\newcommand{\plusc}{{\color{blue} +_c}}
\newcommand{\threshc}{{\color{blue} t_y}}
\newcommand{\succc}[1]{\bluetext{#1 \plusc 1}}

\newcommand{\bluetext}[1]{{\color{blue} #1}}

\newcommand\yield{\textsf{yield}\xspace}
\newcommand\Yield{\textsf{Yield}\xspace}
\newcommand\allows{\textsf{allows}\xspace}

\DeclarePairedDelimiter{\ceil}{\lceil}{\rceil}

\newcommand{\todo}[1]
           {{\par\noindent\small\color{RoyalPurple}
  \framebox{\parbox{\dimexpr\linewidth-2\fboxsep-2\fboxrule}
    {\textbf{TODO:} #1}}}}


\newcommand{\quest}[1]
           {{\par\noindent\small\color{Red}
  \framebox{\parbox{\dimexpr\linewidth-2\fboxsep-2\fboxrule}
    {\textbf{Q:} #1}}}}

\newcommand{\interrupt}[1]{!(#1)}

\newcommand{\fighead}{\textbf}

\newcommand{\lameff}{$\lambda_{\text{eff}}$\xspace}
\newcommand{\lameffrow}{$\lambda_{\text{eff}}^\rho$\xspace}
\newcommand{\feff}{$F_\textrm{eff}$\xspace}
\newcommand{\impeff}{Implicit \lameff}
\newcommand\Frank{\emph{Frank}\xspace}

\newcommand\Cse{\textbf{Case}}

\newcommand{\set}[1]{\{#1\}}
\newcommand{\many}{\overline}
\newcommand{\opt}[1]{#1^?}
\newcommand{\medvert}{\mid}

\newcommand{\sem}[1]{\llbracket{#1}\rrbracket}
\newcommand{\seml}{\left\llbracket}
\newcommand{\semr}{\right\rrbracket}

\newcommand{\mdo}{~\textbf{do}~}
\newcommand{\seq}{~\textbf{;}~}
\newcommand{\assn}[2]{{#1}~\leftarrow~{#2}}
\newcommand{\func}[2]{\text{#1}~{#2}}

\newcommand{\deno}[1]{\sem{#1}\rho}
\newcommand{\denoex}[2]{\sem{#1}#2}
\newcommand{\pc}[1]{\llparenthesis{#1}\rrparenthesis}

\newcommand{\TyVar}{\mathit{Var}}
\newcommand{\dom}{\mathit{dom}}
%\newcommand{\sub}{\subseteq}
\newcommand{\Star}{{\Large$\star$}}

\newcommand{\reducesto}{\longrightarrow}

\newcommand\ba{\begin{array}}
\newcommand\ea{\end{array}}

\newcommand{\bl}{\ba[t]{@{}l@{}}}
\newcommand{\el}{\ea}

\newcommand{\bstack}{\begin{array}[t]{@{}l@{}}}
\newcommand{\estack}{\end{array}}

\newenvironment{equations}{\[\ba{@{}r@{~}c@{~}l@{}}}{\ea\]\ignorespacesafterend}
\newenvironment{eqs}{\ba{@{}r@{~}c@{~}l@{}}}{\ea}

\newenvironment{clauses}{\ba{@{}l@{~}c@{~}l@{}}}{\ea}

\newenvironment{syn}{\ba{@{}l@{~}r@{~}c@{~}l@{}}}{\ea}

\newenvironment{syntax}{\[\ba{@{}l@{~}r@{~}c@{~}l@{}}}{\ea\]\ignorespacesafterend}

\newcommand{\judgeword}[1]{~\mathbf{#1}~}

%\renewcommand{\sig}{\Sigma}
%\renewcommand{\sigs}{\Sigma s}
\newcommand{\sigentails}[1]{\mathbin{[{\text{\scriptsize ${#1}$}}]\hspace{-0.4ex}\text{-\!-}}\,}

%% \newcommand{\sigmodels}[1]{\mathbin{[{\text{\scriptsize ${#1}$}}]\!\mathord{=}}\,}
% \newcommand{\sigentails}[1]{\vdash_{#1}}

\newcommand{\val}[3]  {#1 \vdash {#2} : {#3}}

\newcommand{\rt}[1]{\langle{#1}\rangle}   % returner type

\newcommand{\valg}{\val{\Gamma}}

%% \newcommand{\is}[4]  {#1 \sigentails{#2} {#3} \judgeword{is} {#4}}
%% \newcommand{\isgs}{\is{\Gamma}{\sigs}}

%% \newcommand{\cdoes}[4]{#1 \sigentails{#2} {#3} \judgeword{has} {#4}
%% \newcommand{\cdoesgs}{\cdoes{\Gamma}{\sigs}}


%% some options for rendering bidirectional typing judgements

%% \newcommand{\inferbase}[4]{#2 \mathbin{#1} {#3} \in {#4}}
%% \newcommand{\checkbase}[4]{#2 \mathbin{#1} {#3} \ni {#4}}
%% \newcommand{\patbase}[4]{{#3} \mathbin{:} {#2} \mathbin{#1} {#4}}

\newcommand{\kindcheckbase}[3]{#2 \mathbin{#1} #3} % For well-kindedness of types
\newcommand{\inferbase}[5]{#1; #3 \mathbin{#2} {#4} \Rightarrow {#5}}
\newcommand{\checkbase}[5]{#1; #3 \mathbin{#2} #5 \mathbin{:} #4}
\newcommand{\patbase}[5]{{#1} \vdash {#4} \mathbin{:} {#3} \mathbin{#2} {#5}}
\newcommand{\bindbase}[4]{{#3} \mathbin{:} {#2} \mathbin{#1} {#4}}

%% \newcommand{\inferbase}[4]{#2 \mathbin{#1} {#3} \Rightarrow {#4}}
%% \newcommand{\checkbase}[4]{#2 \mathbin{#1} {#4} \Leftarrow {#3}}
%% \newcommand{\patbase}[4]{{#3} \mathbin{:} {#2} \mathbin{#1} {#4}}

%% \newcommand{\inferbase}[4]{#2 \mathbin{#1} {#3} \uparrow {#4}}
%% \newcommand{\checkbase}[4]{#2 \mathbin{#1} {#4} \downarrow {#3}}
%% \newcommand{\patbase}[4]{{#3} \mathbin{:} {#2} \mathbin{#1} {#4}}

%% \newcommand{\inferbase}[4]{#2 \mathbin{#1} {#3} \judgeword{infers} {#4}}
%% \newcommand{\checkbase}[4]{#2 \mathbin{#1} {#3} \judgeword{checks} {#4}}
%% \newcommand{\patbase}[4]{{#2} \judgeword{matches} {#3} \mathbin{#1} #4}

\newcommand{\makes}[5]{\inferbase{#1}{\sigentails{#3}}{#2}{#4}{#5}}
\newcommand{\has}[5]{\checkbase{#1}{\sigentails{#3}}{#2}{#4}{#5}}
\newcommand{\does}[4]{\checkbase{#1}{\vdash}{#2}{#3}{#4}}
\newcommand{\can}[4]{\makes{\kenv}{#1}{#2}{#3}{#4}}

\newcommand{\effs}[2]{{#1} \judgeword{does} {#2}}


% redefinitions for cbv type system
\newcommand{\kindchecks}[2]{\kindcheckbase{\vdash}{#1}{#2}} % Checks that a type is well-kinded
\newcommand{\infers}{\makes}
\newcommand{\checks}{\has}
\newcommand{\checksdef}{\does}
\newcommand{\matchesc}{\matches}
\newcommand{\matchesck}{\matchesc{\kenv}}

\newcommand{\infersk}{\makes{\kenv}}
\newcommand{\checksk}{\has{\kenv}}
\newcommand{\checksdefk}{\does{\kenv}}

\newcommand{\kindchecksk}{\kindchecks{\kenv}} % Checks that a type is well-kinded
\newcommand{\inferskgs}{\makes{\kenv}{\Gamma}{\sigs}}
\newcommand{\checkskgs}{\has{\kenv}{\Gamma}{\sigs}}
\newcommand{\checksdefkg}{\does{\kenv}{\Gamma}}


\newcommand{\adj}{\Delta}
\newcommand{\adapt}{\Theta}
\newcommand{\ext}{\Xi}
\newcommand{\sigs}{\Sigma}
\newcommand{\sig}{I}

\newcommand{\seed}{\sigma}

\newcommand{\effbox}[1]{[#1]}

\newcommand{\key}[1]{\mathbf{#1}} % keyword
\newcommand{\var}{\mathit}        % local variable or meta variable
\newcommand{\defaultvarname}[0]{x}

\newcommand{\op}{\mathsf}  % operator (command or computation)
\newcommand{\con}{\mathsf} % constructor (type or data)
\newcommand{\inter}{\mathsf} % interface
\newcommand{\str}[1]{\textrm{``#1''}} % string literal


\newcommand{\handleSymbol}{\rightarrow}
\newcommand{\handle}[2]{{#1} \handleSymbol {#2}}

\newcommand{\thunk}[1]{\{{#1}\}}

\newcommand{\force}[1]{{#1}!}

\newcommand{\emptylist}{[]}
\newcommand{\cons}{\mathbin{::}}
\newcommand{\concat}{\,\texttt{++}\,} %mathbin{+\!+}}
%\newcommand{\snoc}{\mathbin{:<}}
\newcommand{\snoc}{\ }


\newcommand{\NN}{\mathbb{N}}

\newcommand{\slab}[1]{(\textrm{#1})}

\newcommand{\ev}{E}
\newcommand{\evd}{\varepsilon}

\newcommand{\effin}[1]{\langle {#1} \rangle}
\newcommand{\effout}[1]{[{#1}]}

\newcommand{\nowt}{\emptyset}
\newcommand{\id}{\iota}
\newcommand{\pid}{\var{s}} % Pattern identity variable

\newcommand{\EC}{\mathcal{E}}
\newcommand{\EF}{\mathcal{F}}
\newcommand{\PC}{\mathcal{P}} % Syntactic phrase class for af operation
\newcommand{\venv}{\theta}

\newcommand{\freeze}{\ceil}

\newcommand{\uc}{\mathord{\downarrow}}
\newcommand{\cu}{\mathord{\uparrow}}

\newcommand{\redtou}{\leadsto_{\mathrm{u}}}
\newcommand{\redtoc}{\leadsto_{\mathrm{c}}}
\newcommand{\stepsto}{\longrightarrow}

\newcommand{\stepstou}{\longrightarrow_{\mathrm{u}}}
\newcommand{\stepstoc}{\longrightarrow_{\mathrm{c}}}

\newcommand{\sigat}{\mathbin{@}}

\newcommand{\meta}{\mathsf}
\newcommand{\level}{\meta{level}}
\newcommand{\af}{\meta{af}}
\newcommand{\handles}{~\meta{handles}~}

\newcommand{\poised}{~\meta{poisedfor}~}
\newcommand{\insts}{\meta{inst}}
\newcommand{\remap}{\meta{remap}}

\newcommand{\sigyields}[1]
           {\mathbin{\text{-\!-\!}[{\text{\scriptsize ${#1}$}}]\,}}

\newcommand{\matches}[5]{\patbase{#1}{\sigyields{#4}}{#2}{#3}{#5}}
\newcommand{\matchesv}[4]{\patbase{#1}{\dashv}{#2}{#3}{#4}}
\newcommand{\matchesvk}{\matchesv{\kenv}}

\newcommand{\bindsv}[4]{\bindbase{\dashv}{#2 \leftarrow #3}{#1}{#4}}
\newcommand{\bindsc}[5]{\bindbase{\sigyields{#4}}{#2 \leftarrow #3}{#1}{#5}}

\newcommand{\letin}[4][\defaultvarname]
           {\key{let}\;{#1}:{#2}={#3}\;\key{in}\;{#4}}
\newcommand{\letxin}[3][\defaultvarname]
           {\key{let}\;{#1}={#2}\;\key{in}\;{#3}}
\newcommand{\letrec}[4][f]{\key{letrec}~\many{{#1}:{#2} = {#3}}~\key{in}~{#4}}
\newcommand{\letrecU}[3][f]{\key{letrec}~\many{{#1} = {#2}}~\key{in}~{#3}}
\newcommand{\Gt}{\theta} % Substitution meta variable
\newcommand{\submap}[2]{{{#1}\vDash{#2}}}
\newcommand{\sub}[4]{#1 \vdash {{#2}:\submap{{#3}}{{#4}}}}
\newcommand{\subk}{\sub{\kenv}}
\newcommand{\subext}[2]{{{#1}{#2}}}
\newcommand{\subst}[3][\defaultvarname]{{#2}[{#3}/{#1}]}

% Frank letrec substitution
\newcommand{\recsub}[5][f]
      {[\many{\cu (\thunk{\many{\many{#2}\mapsto\letrec[{#1}]{#5}{#3}{#4}}}
            : {#5})/{#1}}]}


%%%% START inference rule system for action of adjustment on ability %%%%
\newcommand{\semi}{;}
\newcommand{\kenv}{\Phi}  % kind environment
\newcommand{\kenva}{\Psi} % another kind environment
%% \newcommand{\kenv}{\mathcal{T}} % kind environment
\newcommand{\ienv}{\Omega} % Instance environment
\newcommand{\adjact}[3]{{#1}\vdash{#2}\dashv{#3}}
\newcommand{\adpcom}[5]{{{#1}\vdash{#2}({#3} \to {#4})\dashv{#5}}}
\newcommand{\itrbnd}[5]{{{#1}\vdash{#2}:{#3}\dashv{#4}\semi{#5}}}
% \newcommand{\wf}[2]{{{#1}\vdash{#2}}}
\newcommand{\itrinst}[4]{{{#1}\vdash{#2}:{#3}\dashv{#4}}}

%%% END inference rule system for action of adjustment on ability %%%%%

% Untyped machine letrec substitution
\newcommand{\recsubst}[5]
 {{#1}[\many{(\thunk{\many{\many{#2} \mapsto \letrecU{#3}{#4}}}:{#5})/f}]}

%% Abstract machine commands
% Typing
\newcommand{\HAbs}[2]{{{#1}\to{#2}}}

\newcommand{\fail}{\textbf{fail}}

%% Translation function: Frank Terms to Untyped A-Normal Form
\newcommand{\UANF}[1]{{\llbracket{{#1}}\rrbracket}}

% Terms
\newcommand{\mtrns}[3][]{{#2} & \Rightarrow^{#1} & {#3}} % For array env
\newcommand{\mtrnsR}[3][]{{#2}\Rightarrow^{#1}{#3}}

\newcommand{\confg}[2]{{\langle{{#1}},{{#2}}\rangle}}
\newcommand{\term}[3]
           {{\langle{{#1}},{{#2}}\rangle\downarrow{#3}}}

\newcommand{\admin}[2]{{\langle{{#1}}\mid{{#2}}\rangle}}
\newcommand{\mat}[3]
           {{\langle{{#1}}\mid{{#2}}\mid{{#3}}\rangle}}
\newcommand{\matc}[5]
        {{\langle{{#1}}\mid{{#2}}\mid{{#3}}\mid{{#4}}\mid{{#5}}\rangle}}

\newcommand{\msub}[3][\defaultvarname]{{#2}[{#1}\mapsto{#3}]}

\newcommand{\FHan}[4][\many{\effin{\adj}}]{{({#2}:{#1},{#3}\mid{#4})}}
\newcommand{\FSeq}[2][\defaultvarname]
           {{({#1}.{#2})}}
\newcommand{\SCons}[2]{{{#1}\circ{#2}}}


\newcommand{\HSHan}[5][C]{{{#2}\circ({#3},{#1},{#4}\mid{#5})}}
\newcommand{\HSSeq}[4][\defaultvarname]
           {{#2}\circ({#1}:{#3}.{#4})}
\newcommand{\HSCons}[2]{{{#1}\circ{#2}}}
\newcommand{\NF}[2]{{{#1}~\star~{#2}}}

\newcommand{\evalto}{\Longrightarrow}


\newcommand{\para}[1]{\paragraph{#1.}}

\newcommand{\gor}{\mid}
\newcommand{\pipe}{\texttt{|}}


%%%%%%%%%%%%%%%%%%%%%%%%%%%%%%%%%%%%%%%%%%%%%%%%%
%% End of inference rules typesetting business
%%%%%%%%%%%%%%%%%%%%%%%%%%%%%%%%%%%%%%%%%%%%%%%%%






%%%%%%%%%%%%%%%%%%%%%%%%%%%%%%%%%%%%%%%%
%% Start of main text
%%%%%%%%%%%%%%%%%%%%%%%%%%%%%%%%%%%%%%%%

\begin{document}
\begin{preliminary}

\title{This is the Project Title}

\author{Your Name}

\abstract{
Formal development of Frank.
}

\maketitle

%% \section*{Acknowledgements}
%% thanks to tony soprano

%% \tableofcontents

\end{preliminary}




\chapter{Formalisation of Frank}


\begin{figure}  %\figrule
\[
\ba{@{}c@{}}
\ba{@{}c@{\quad\quad}c@{}}
\begin{syn}
  \slab{data types}            & D \\
  \slab{value type variables}  & X \\
  \slab{effect type variables} & E \\
  \slab{value types}           & A, B   &::= & D~\overline{R} \\
                               &        &\gor& \thunk{C} \gor X \\
  \slab{computation types}     & C      &::= & \many{T \to}~G \\
  \slab{argument types}        & T      &::= & \effin{\adj}A \\
  \slab{return types}          & G      &::= & \effout{\sigs}A \\

  \slab{type binders}          & Z      &::= & X \gor [E]\\
  \slab{type arguments}        & R      &::= & A \gor [\Sigma]\\
  \slab{polytypes}             & P      &::= & \forall \overline{Z}.A \\
\end{syn}
&
\begin{syn}
  \slab{interfaces}           & I \\
  \slab{term variables}       & x, y, z, f \\
  \slab{instance variables}   & \pid, a, b, c \\
  \slab{seeds}                & \seed  &::= & \nowt \gor \ev \\
  \slab{abilities}            & \sigs  &::= & \seed\pipe\ext \\
  \slab{extensions}           & \ext   &::= & \id \gor \ext, \sig~\many{R} \\
  \slab{adaptors}             & \adapt &::= & \id \gor \adapt, \sig(S \to S') \\
  \slab{adjustments}          & \adj   &::= & \adapt\pipe\ext \\
  \slab{instance patterns}    & S      &::= & \pid \gor S \snoc a \\
  \slab{kind environments}    & \kenv,
                                \kenva &::= & \cdot \gor \kenv, Z \\
  \slab{type environments}    & \Gamma &::= & \cdot \gor \Gamma, x:A %\\
%                              &        &    & \hphantom{\cdot}
                                              \gor \Gamma, f:P\\
 \slab{instance environments} & \ienv  &::= & \pid:\sigs \gor \ienv, a:\sig~\many{R}\\
\end{syn} \\
\ea \\
\ea
\]
\\[0.25cm]

\caption{Types}
\label{fig:types}
%\figrule
\end{figure}




\begin{figure} %\figrule
\begin{syntax}
  %% \slab{monomorphic term variables} & x, y, z \\
  %% \slab{polymorphic term variables} & f \\
  \slab{constructors}               & k \\
  \slab{commands}                   & c \\
  \slab{uses}                 & m      &::= &
     x \gor f~\many{R} \gor m~\many{n} \gor \cu(n:A) \\
  \slab{constructions}        & n      &::= &
    \uc m \gor k~\many{n} \gor c~\many{R}~\many{n} \gor \thunk{e} \\
                              &        &\gor& \key{let}~f : P = n~\key{in}~n'
                                   \gor
                                   \key{letrec}~\many{f : P = e}~\key{in}~n \\
                              &        &\gor&  \effin{\adapt}~n \\
  \slab{computations}         & e      &::=& \many{\many{r} \mapsto n}
  \\
  \slab{computation patterns} & r      &::=& p
                                        \gor \effin{\handle{c~\many{p}\,}{z}}
                                        \gor \effin{x} \\
  \slab{value patterns}       & p      &::=& k~\many{p} \gor x        \\
\end{syntax}
\\[0.25cm]
%\textit{with} term variables $x$, $y$, $z$, polymorphic term variables $f$, constructors $k$, commands $c$\\[0.25cm]
\caption{Terms}
\label{fig:terms}
% \figrule
\end{figure}



\begin{figure} % \figrule
\flushleft
%% $\boxed{\kindchecksk{R}}$
%% \begin{mathpar}
%% \inferrule[K-Val]
%%   {\TyVar(A) \subseteq \kenv}
%%   {\kindchecksk{A}}

%% \inferrule[K-Eff]
%%   {\TyVar(\sigs) \subseteq \kenv}
%%   {\kindchecksk{[\sigs]}}
%% %
%% \end{mathpar}

$\boxed{\infersk{\Gamma}{\sigs}{m}{A}}$
\begin{mathpar}
\inferrule[T-Var]
  {
   x:A \in \Gamma}
  {\inferskgs{x}{A}}

\inferrule[T-PolyVar]
  {% \kindchecks{\kenv, \many{Z}}{A}\\
   \kindchecksk{\many{R}} \\
   f:\forall \many{Z}.A \in \Gamma}
  {\inferskgs{f~\many{R}}{A[\many{R}/\many{Z}]}}
\\
\inferrule[T-App]
  {\sigs' = \sigs \\
   (\adjact{\sigs}{\adj_i}{\sigs'_i})_i \\\\
   \inferskgs{m}{\thunk{\many{\effin{\adj}A \to}~ \effout{\sigs'}B}} \\
   (\checksk{\Gamma}{\sigs'_i}{A_i}{n_i})_i}
  {\infersk{\Gamma}{\sigs}{m~\many{n}}{B}}

\inferrule[T-Ascribe]
  {\checkskgs{A}{n}}
  {\inferskgs{\cu (n : A)}{A}}
%
\end{mathpar}

$\boxed{\checksk{\Gamma}{\sigs}{A}{n}}$
\begin{mathpar}
\inferrule[T-Switch]
  {\inferskgs{m}{A} \\ A = B}
  {\checkskgs{B}{\uc m}}

\inferrule[T-Data]
  {%(\kindchecksk{R_i})_i\\
   k~\many{A} \in D~\many{R} \\
   (\checkskgs{A_j}{n_j})_j}
  {\checkskgs{D~\many{R}}{k~\many{n}}}

\inferrule[T-Command]
  {\kindchecksk{\many{R}} \\
   c : \forall \many{Z}.\many{A \to}~ B \in \sigs \\
   (\checkskgs{A_j[\many{R}/\many{Z}]}{n_j})_j}
  {\checkskgs{B[\many{R}/\many{Z}]}{c~\many{R}~\many{n}}}

\inferrule[T-Thunk]
  {\checksdefkg{C}{e}}
  {\checkskgs{\thunk{C}}{\thunk{e}}}

\inferrule[T-Let]
  {P = \forall \many{Z}.A \\\\
   \checkbase{\kenv, \many{Z}}{\sigentails{\emptyset}}{\Gamma}{A}{n} \\
   \checksk{\Gamma, f : P}{\sigs}{B}{n'}}
  {\checkskgs{B}{\key{let}~f : P = n~\key{in}~n'}}

\inferrule[T-LetRec]
  {(P_i = \forall \many{Z}_i.\thunk{C_i})_i \\\\
   (\checkbase{\kenv, \many{Z}_i}{\vdash}{\Gamma, \many{f : P}}{C}{e_i})_i\\
   \checksk{\Gamma, \many{f : P}}{\sigs}{B}{n}}
  {\checkskgs{B}{\key{letrec}~\many{f : P = e}~\key{in}~n}}

\inferrule[T-Adapt]
  {\adjact{\sigs}{\adapt}{\sigs'} \\ \checksk{\Gamma}{\sigs'}{A}{n}}
  {\checkskgs{A}{\effin{\adapt}~n}}
\end{mathpar}

$\boxed{\checksdefkg{C}{e}}$
\begin{mathpar}
\inferrule[T-Comp]
  {(\matchesck{T_j}{r_{i,j}}{\sigs}{\exists \kenva_{i,j}.\Gamma'_{i,j}})_{i,j} \\
   (\checks{\kenv, (\kenva_{i,j})_j}{\Gamma, (\Gamma'_{i,j})_j}{\sigs}{B}{n_i})_i \\
   ((r_{i,j})_{i} \text{ covers } T_j)_j}
  {\checksdefkg{(T_j \to)_j~\effout{\sigs}B}{((r_{i,j})_j \mapsto n_i)_i}}
\end{mathpar}
\caption{Term Typing Rules}
\label{fig:term-typing}
% \figrule
\end{figure}



\begin{figure}[t]
%% \figrule
\begin{syntax}
\slab{uses}                    & m   &::= & \dots \mid \freeze{\EC[c~\many{R}~\many{w}]} \\
\slab{constructions}           & n   &::= & \dots \mid \freeze{\EC[c~\many{R}~\many{w}]} \\
\slab{use values}              & u   &::= & x \gor f~\many{R} \gor \cu (v : A) \\
\slab{non-use values}          & v   &::= & k~\many{w} \gor \thunk{e} \\
\slab{construction values}     & w   &::= & \uc u \gor v \\
\slab{normal forms}            & t   &::= & w \gor \freeze{\EC[c~\many{R}~\many{w}]} \\
\slab{evaluation frames}       & \EF &::= & [~]~\many{n}
                                      \gor  u~(\many{t},[~],\many{n})
                                      \gor  \cu([~]:A) \\
                               &     &\gor& \uc [~]
                                      \gor  k~(\many{w},[~],\many{n})
                                      \gor  c~\many{R}~(\many{w},[~],\many{n}) \\
                               &     &\gor& \key{let}~f: P = [~]~\key{in}~n
                                      \gor \effin{\adapt}~[~] \\
\slab{evaluation contexts}     & \EC &::= & [~] \gor \EF[\EC] \\
\end{syntax}
\caption{Runtime Syntax}
\label{fig:runtime-syntax}
%% \figrule
\end{figure}



\begin{figure}[t]
%% \figrule
\flushleft
$\boxed{\inferskgs{m}{A}}$ \quad $\boxed{\checkskgs{A}{n}}$
\begin{mathpar}
\inferrule[T-Freeze-Use]
  {\neg(\EC \handles c) \\
   \inferskgs{\EC[c~\many{R}~\many{w}]}{A}}
  {\inferskgs{\freeze{\EC[c~\many{R}~\many{w}]}}{A}}

\inferrule[T-Freeze-Cons]
  {\neg(\EC \handles c) \\
   \checkskgs{A}{\EC[c~\many{R}~\many{w}]}}
  {\checkskgs{A}{\freeze{\EC[c~\many{R}~\many{w}]}}}
\end{mathpar}
\caption{Frozen Commands}
\label{fig:frozen-typing}
%% \figrule
\end{figure}


\begin{figure}
%% \figrule
\flushleft
$\boxed{m \redtou m'} \quad \boxed{n \redtoc n'} \quad \boxed{m
    \stepstou m'} \quad \boxed{n \stepstoc n'}$
\begin{mathpar}
\inferrule[R-Handle]
  {% The type system should enforce this!
   %(\handles{\adj_j}{t_j})_j \\
   k = \min_i\,\{i \mid \exists \many{\venv}.(\bindsc{r_{i,j}}{\effin{\adj_j} A_j}{t_j}{\sigs}{\venv_j})_{j}\} \\
   (\bindsc{r_{k,j}}{\effin{\adj_j} A_j}{t_j}{\sigs}{\venv_j})_{j}}
   %% \forall i < k.\exists j.r_{i, j} \# t_j}
   %% l \text{~is minimal}}
  {\cu (\thunk{((r_{i,j})_j \to n_i)_i} : \thunk{\many{\effin{\adj} A \to}~\effout{\sigs}B})~\many{t} \redtou \cu ((\many{\venv}(n_k) : B)}

\inferrule[R-Ascribe-Use]
  { }
  {\cu(\uc u:A) \redtou u}

\inferrule[R-Ascribe-Cons]
  { }
  {\uc \cu (w : A) \redtoc w}

\inferrule[R-Let]
  { }
  {\key{let}~f:P = w~\key{in}~n \redtoc n[\cu (w : P)/f]}

\inferrule[R-LetRec]
  {\many{e = \many{\many{r} \to n}}}
  {%\vphantom{\many{\many{\many{\many{f}}}}}
   \key{letrec}~\many{f:P = e}~\key{in}~n' \redtoc
    n'[\many{\cu (\thunk{\many{\many{r} \to \key{letrec}~\many{f:P = e}~\key{in}~n}}: P)/f}]}

\inferrule[R-Adapt]
  { }
  {\effin{\adapt}~w \redtoc w}

\inferrule[R-Freeze-Comm]
  { }
  {c~\many{R}~\many{w} \redtoc \freeze{c~\many{R}~\many{w}}}\\

\inferrule[R-Freeze-Frame-Use]
  {\neg(\EF[\EC] \handles c)}
  {\EF[\freeze{\EC[c~\many{R}~\many{w}]}] \redtou \freeze{\EF[\EC[c~\many{R}~\many{w}]]}}

\inferrule[R-Freeze-Frame-Cons]
  {\neg(\EF[\EC] \handles c)}
  {\EF[\freeze{\EC[c~\many{R}~\many{w}]}] \redtoc \freeze{\EF[\EC[c~\many{R}~\many{w}]]}}

\\
\inferrule[R-Lift-UU]
  {m \redtou m'}
  {\EC[m] \stepstou \EC[m']}

\inferrule[R-Lift-UC]
  {m \redtou m'}
  {\EC[m] \stepstoc \EC[m']}

\inferrule[R-Lift-CU]
  {n \redtoc n'}
  {\EC[n] \stepstou \EC[n']}

\inferrule[R-Lift-CC]
  {n \redtoc n'}
  {\EC[n] \stepstoc \EC[n']}
\end{mathpar}

\caption{Operational Semantics}
\label{fig:small-step}
%% \figrule
\end{figure}


\begin{figure}[t]
%% \figrule
\flushleft
%\textit{Comp. pattern $r$ for $\langle \Delta \rangle A$ matches $u$ under
%amb. $\venv$ and binds $\venv$.}
$\boxed{\bindsc{r}{T}{t}{\sigs}{\venv}}$
\begin{mathpar}

\inferrule[B-Value]
  {\adjact{\sigs}{\adj}{\sigs'} \\\\ \bindsv{p}{A}{w}{\venv}}
  {\bindsc{p}{\effin{\adj}A}{w}{\sigs}{\venv}}

  \inferrule[B-Request]
    {%I~\many{R} \in \ext \\ %\capturesI{\Delta}{I}{\iota}\\
    \adjact{\sigs}{\adj}{\sigs'} \\
    \EC \poised c \\\\
    \adj = \adapt\pipe\ext \\
    c : \forall \many{Z}. \many{B \to}~B' \in \ext \\
    (\bindsv{p_i}{B_i}{w_i}{\venv_i})_i}
    {\bindsc{\effin{c~\many{p} \to z}}{\effin{\adj}A}
    {\freeze{\EC[c~\many{R}~\many{w}]}}{\sigs}{\many{\venv}[\cu (\thunk{x \mapsto \EC[x]} : \thunk{B' \to \effout{\sigs'}A})/z]}}

\inferrule[B-CatchAll-Value]
  {\adjact{\sigs}{\adj}{\sigs'}}
  {\bindsc{\effin{x}}{\effin{\adj}A}{w}{\sigs}{[\cu (\thunk{w}\mathord{:}\thunk{\effout{\sigs'}A})/x]}}
\\
\inferrule[B-CatchAll-Request]
  {
  \adjact{\sigs}{\adj}{\sigs'} \\
  \EC \poised c \\\\
  \adj = \adapt\pipe\ext \\
  c : \forall \many{Z}. \many{B \to}~B' \in \ext
  }
  {\bindsc{\effin{x}}{\effin{\adj}A}
  {\freeze{\EC[c~\many{R}~\many{w}]}}
  {\sigs}
  {[\cu (\thunk{\freeze{\EC[c~\many{R}~\many{w}]}}\mathord{:}\thunk{\effout{\sigs'}A})/x]}}

\end{mathpar}


%~~ \textit{Value pattern $p$ for type $A$ matches $w$ and binds $\venv$.}
%
$\boxed{\bindsv{p}{A}{w}{\venv}}$
\begin{mathpar}

\inferrule[B-Var]
  { }
  {\bindsv{x}{A}{w}{[\cu (w : A)/x]}}

\inferrule[B-Data]
  {k~\many{A} \in D~\many{R} \\
   (\bindsv{p_i}{A_i}{w_i}{\venv_i})_i}
 {\bindsv{k~\many{p}}{D~\many{R}}{k~\many{w}}{\many{\venv}}}
\end{mathpar}

\caption{Pattern Binding}
\label{fig:pattern-binding}
%% \figrule
\end{figure}



\chapter{Arbitrary Thread Interruption}


\section{Relaxing Catches}\label{sec:relaxingcatches}


\begin{figure}[h]
%% \figrule
%% \flushleft
%% \centering
%\textit{Comp. pattern $r$ for $\langle \Delta \rangle A$ matches $u$ under
%amb. $\venv$ and binds $\venv$.}
\begin{mathpar}

\inferrule[B-CatchAll-Request-Loose]
  {
  \adjact{\sigs}{\adj}{\sigs'} \\
  %% \EC \poised c \\\\
  %% \adj = \adapt\pipe\ext \\
  %% c : \forall \many{Z}. \many{B \to}~B' \in \ext
  }
  {\bindsc{\effin{x}}{\effin{\adj}A}
  {\freeze{\EC[c~\many{R}~\many{w}]}}
  {\sigs}
  {[\cu (\thunk{\freeze{\EC[c~\many{R}~\many{w}]}}\mathord{:}\thunk{\effout{\sigs'}A})/x]}}

\end{mathpar}

\caption{Updated \textsc{B-CatchAll-Request}}
\label{fig:loose-catchall-request}
%% \figrule
\end{figure}


\section{Interrupting Arbitrary Terms}


\begin{figure}[h]
%% \figrule
\begin{syntax}
\slab{uses}                    & m   &::= & \dots \mid \freeze{\EC[c~\many{R}~\many{w}]} \\
\slab{constructions}           & n   &::= & \dots \mid \freeze{\EC[c~\many{R}~\many{w}]} \\
\slab{use values}              & u   &::= & x \gor f~\many{R} \gor \cu (v : A) \\
\slab{non-use values}          & v   &::= & k~\many{w} \gor \thunk{e} \\
\slab{construction values}     & w   &::= & \uc u \gor v \\
\slab{normal forms}            & t   &::= & w \gor \freeze{\EC[c~\many{R}~\many{w}]} \gor \interrupt{m}\\
\slab{evaluation frames}       & \EF &::= & [~]~\many{n}
                                      \gor  u~(\many{t},[~],\many{n})
                                      \gor  \cu([~]:A) \\
                               &     &\gor& \uc [~]
                                      \gor  k~(\many{w},[~],\many{n})
                                      \gor  c~\many{R}~(\many{w},[~],\many{n}) \\
                               &     &\gor& \key{let}~f: P = [~]~\key{in}~n
                                      \gor \effin{\adapt}~[~] \\
\slab{evaluation contexts}     & \EC &::= & [~] \gor \EF[\EC] \\
\end{syntax}
\caption{Runtime Syntax, Updated with Suspension of Uses}
\label{fig:runtime-syntax-suspend}
%% \figrule
\end{figure}


\begin{figure}[h]
%% \figrule
\flushleft
%\textit{Comp. pattern $r$ for $\langle \Delta \rangle A$ matches $u$ under
%amb. $\venv$ and binds $\venv$.}
\begin{mathpar}
\inferrule[B-CatchAll-Interrupt]
  {\adjact{\sigs}{\adj}{\sigs'}}
  {\bindsc{\effin{x}}{\effin{\adj}A}{\interrupt{m}}{\sigs}{[\cu (\thunk{m}\mathord{:}\thunk{\effout{\sigs'}A})/x]}}
\end{mathpar}

\caption{Catching Interrupts rule. }
\label{fig:catchall-interrupt}
%% \figrule
\end{figure}

\begin{figure}[h]
%% \figrule
\flushleft
\begin{mathpar}
%% \\
\inferrule[R-Interrupt]
  { }
  {m \redtou \interrupt{m}}
%% \\
\end{mathpar}

\caption{Use interruption rule}
\label{fig:r-interrupt}
%% \figrule
\end{figure}

\newpage
\section{Freezing}

\hl{Another} way of doing it is to just let any use become frozen, in the same way as
commands become frozen once invoked.


\begin{figure}[h]
%% \figrule
\flushleft
$\boxed{m \redtou m'} \quad \boxed{n \redtoc n'} \quad \boxed{m
    \stepstou m'} \quad \boxed{n \stepstoc n'}$
\begin{mathpar}

\inferrule[R-Freeze-Use]
  {  }
  { m \redtou \freeze{m} }

\inferrule[R-Freeze-Comm]
  { }
  {c~\many{R}~\many{w} \redtoc \freeze{c~\many{R}~\many{w}}}\\

\inferrule[R-Freeze-Frame-Use]
  { }
  { \EF[\freeze{m}] \redtou \freeze{\EF[m]} }

\inferrule[R-Freeze-Frame-Cons]
  { }
  { \EF[\freeze{m}] \redtoc \freeze{\EF[m]} }

\inferrule[R-Freeze-Frame-Use]
  {\neg(\EF[\EC] \handles c)}
  {\EF[\freeze{\EC[c~\many{R}~\many{w}]}] \redtou \freeze{\EF[\EC[c~\many{R}~\many{w}]]}}

\inferrule[R-Freeze-Frame-Cons]
  {\neg(\EF[\EC] \handles c)}
  {\EF[\freeze{\EC[c~\many{R}~\many{w}]}] \redtoc \freeze{\EF[\EC[c~\many{R}~\many{w}]]}}

\end{mathpar}

\caption{Updated Freezing}
\label{fig:Freezing}
%% \figrule
\end{figure}

\todo{Do we need the 'hoisting' rules for general suspended terms?}


\begin{figure}[h]
%% \figrule
\begin{syntax}
\slab{uses}                    & m   &::= & {\dots} \gor
\freeze{\EC[c~\many{R}~\many{w}]} \gor \highlight{\freeze{m}} \\
\slab{constructions}           & n   &::= & \dots \gor
\freeze{\EC[c~\many{R}~\many{w}]} \gor \highlight{\freeze{m}} \\
\slab{use values}              & u   &::= & x \gor f~\many{R} \gor \cu (v : A) \\
\slab{non-use values}          & v   &::= & k~\many{w} \gor \thunk{e} \\
\slab{construction values}     & w   &::= & \uc u \gor v \\
\slab{normal forms}            & t   &::= & w \gor \freeze{\EC[c~\many{R}~\many{w}]} \gor \highlight{\freeze{m}}\\
\slab{evaluation frames}       & \EF &::= & [~]~\many{n}
                                      \gor  u~(\many{t},[~],\many{n})
                                      \gor  \cu([~]:A) \\
                               &     &\gor& \uc [~]
                                      \gor  k~(\many{w},[~],\many{n})
                                      \gor  c~\many{R}~(\many{w},[~],\many{n}) \\
                               &     &\gor& \key{let}~f: P = [~]~\key{in}~n
                                      \gor \effin{\adapt}~[~] \\
\slab{evaluation contexts}     & \EC &::= & [~] \gor \EF[\EC] \\
\end{syntax}
\caption{Runtime Syntax, Updated with Freezing of Uses}
\label{fig:runtime-syntax-freeze}
%% \figrule
\end{figure}

\begin{figure}[h]
%% \figrule
\flushleft
%\textit{Comp. pattern $r$ for $\langle \Delta \rangle A$ matches $u$ under
%amb. $\venv$ and binds $\venv$.}
\begin{mathpar}
\inferrule[B-CatchAll-Interrupt]
  {\adjact{\sigs}{\adj}{\sigs'}}
  {\bindsc{\effin{x}}{\effin{\adj}A}{\freeze{m}}{\sigs}{[\cu (\thunk{m}\mathord{:}\thunk{\effout{\sigs'}A})/x]}}
\end{mathpar}

\caption{Catching Frozen Terms rule.}}
\label{fig:catchall-frozen}
%% \figrule
\end{figure}

\section{Yielding}
\label{sec:yielding}


\begin{figure}[h]
%% \figrule
%% \flushleft
$\boxed{m \redtou m'} $
\begin{mathpar}

%% \inferrule[R-Yield]
%%   {
%%     \sigs = \seed\pipe\ext \\
%%     \textsf{Yield} \in \ext
%%   }
%%   { \cu({n : [\sigs] A})
%%     \redtou
%%     \cu(
%%     \key{let}~f_n : \{\textsf{Unit} \to [\sigs]A\} = \{\_ \mapsto n\}~\key{in}~
%%     (\key{let}~y : [\sigs]\textsf{Unit} = \textsf{yield}!~\key{in}~
%%     f_n~y) : [\sigs] A)
%%   }

%%   \\
%% %% \inferrule[R-Yield]
%% %%   {
%% %%     \sigs = \seed\pipe\ext \\
%% %%     \textsf{Yield} \in \ext
%% %%   }
%% %%   { \cu({\thunk{n} : \thunk{[\sigs] A}})
%% %%     \redtou
%% %%     \cu(
%% %%     \key{let}~f_n : \{\textsf{Unit} \to \thunk{[\sigs]A}\} = \{\_ \mapsto
%% %%     \thunk{n}\}~\key{in}~ \\
%% %%     (\key{let}~\thunk{y} : \thunk{[\sigs]\textsf{Unit}} = \textsf{yield}~\key{in}~
%% %%     f_n~y!) : \thunk{[\sigs] A})
%% %%   }

%% %% \\
\inferrule[R-Yield]
  {
    \sigs = \seed\pipe\ext \\
    \textsf{Yield} \in \ext
  }
  {
    {\begin{array}{rl}
    \cu({\thunk{n} : \thunk{[\sigs] A}})
    \redtou
    \cu(&\key{let}~f_n : \{\textsf{Unit} \to [\sigs]A\} = \{\_ \mapsto
    n\}~\key{in}~ \\
    &\key{let}~y : \thunk{[\sigs]\textsf{Unit}} = \thunk{\textsf{yield!}}~\key{in}~
    \thunk{f_n~(\uc{(y!)})} : \thunk{[\sigs] A})
      \end{array}
      }
  }

\\
\inferrule[R-Yield-EF]
          { \EF[\EC]~\allows~\textsf{Yield} }
          { \EF[\EC[m]] \redtou \EF[\EC[\textsf{yield!}; m]] }
\\

\end{mathpar}

\caption{Inserting Yields
  \todo{In R-Yield-EF we have to let terms inside eval
    CTXs be also uses. Does this not mean that we can then put any term in
    either one? Is that a problem?}
  }
\label{fig:insert-yield}
%% \figrule
\end{figure}

%% We explain \textsc{R-Yield} here. Given a suspended construction $\thunk{n}$,
%% which has been ascribed the type $\thunk{[\sigs]A}$, we produce a new function
%% $f_n$ which takes a value of type $\textsf{Unit}$ and returns the original term
%% $n$, and a thunk $y$ which is just $\textsf{yield!}$ suspended. Note that we
%% write $\textsf{yield}!$ for the 0-ary invocation of the command
%% $\textsf{yield}$. We then apply this constructed function $f_n$ to $y!$; note
%% that because the interface $\textsf{Yield}$ is a member of the ability $\sigs$
%% this is well-typed.

Note that \textsc{R-Yield-EF} relies on the predicate
$\EF[\EC]~\allows~c$. This states that the \textsf{Yield}
interface is in the ability of the term in the evaluation context.

For any frame apart from argument frames,
$\EF[\EC]~\allows~c = \textsf{false}$. In this case, it is defined as
follows;

%% \begin{equations}
%%   \cu (v : \thunk{\many{\effin{\adj} A
%%       \to}~\effout{\sigs}B})~(\many{t},[~],\many{n})~\allows~c =
%%           \left\{ \ba{@{}l@{\quad}l@{}}
%%           \ext~\allows~c & \text{if } |\many{n}| = 0~\text{ where } \sigs =~\seed\pipe\ext \\
%%           \textsf{false} & \text{otherwise}
%%           \right.

%% \end{equations}
\begin{equations}
  \cu (v : \thunk{\many{\effin{\adj} A
      \to}~\effout{\sigs}B})~(\many{t},[~],\many{n})~\allows~c =
  \sigs'~\allows~c ~\text{ where } \adjact{\sigs}{\adj_{|\many{t}|}}{\sigs'}    \\
\end{equations}

\todo{What happens for 0-ary constructors? I don't know if it's exactly how I
  think it is.}

\noindent For an ability $\sigs = \seed\pipe\ext$, $\sigs~\allows~c$ is true if
$c \in I$ for some $I \in \ext$.

%% \noindent For an extension $\ext$, the \allows predicate is defined as

%% \begin{equations}
%%   \id~\allows~c &=& \textsf{false} \\
%%   (\ext, \sig~\many{R})~\allows~c &=&
%%       \left\{ \ba{@{}l@{\quad}l@{}}
%%       \textsf{true} & \text{if } c\in\sig \\
%%       \ext~\allows~c & \text{otherwise}
%%       \right.
%% \end{equations}

Informally, $\EC~\allows~c$ is true when $\EC$ is a handler where the command
$c$ is a member of an interface in its ability when modified by the adaptor at
the corresponding position.

\todo{Clean this up}

%% Informally, $\EF[\EC]~\allows~c$ is true when $\EF[\EC]$ is a handler
%% with the command $c$ as a member of an interface in its ability, \emph{and} all
%% of the arguments have been evaluated (this is what the $|\many{n}| = 0$
%% constraint expresses).

We also make use of an auxilary combinator $\_ ; \_$. This is the traditional
sequential composition operator, where both arguments are evaluated and hte
result of the second one is returned. In the context of \textsc{R-Yield-EF} this
means we will perform the \textsf{yield} effect and then the use $m$, but
discard the result from \textsf{yield}.

%% \begin{equations}
%% \af(c, \id) &=& s \to s \\
%% \af(c, (\ext, \sig~\many{R})) &=&\left\{
%%                                     \ba{@{}l@{\quad}l@{}}
%%                                     s \to s \snoc \bot, & \text{if } c \in I \\
%%                                     \af(c, \ext), & \text{otherwise} \\
%%                                     \ea
%%                                   \right.
%% \end{equations}



\section{Counting}

In practise we count up through the amount of \textsc{R-Handle} rules we apply
and only insert the yield when this count exceeds a threshold value $t_y$.

So we supplement the operational semantics with a \emph{counter} $c_y$, so
that our transitions are e.g. $m; c_y \redtou m'; {c_y}'$. We adopt the
convention that, when the counter is not mentioned in a transition\footnote{That
is, the transition is of the form $m \redtou m'$.}, the counter stays the same.
Hence e.g. $m \redtou m'$ desugars to $m; c_y \redtou m'; {c_y}$.

So to get our counting semantics we just need to supplement Figure
\cite{fig:operational-semantics} with the updated rule in


\begin{figure}
%% \figrule
\flushleft
$\boxed{m \redtou m'} \quad \boxed{n \redtoc n'} \quad \boxed{m
    \stepstou m'} \quad \boxed{n \stepstoc n'}$
\begin{mathpar}
\inferrule[R-Handle]
  {% The type system should enforce this!
   %(\handles{\adj_j}{t_j})_j \\
   k = \min_i\,\{i \mid \exists \many{\venv}.(\bindsc{r_{i,j}}{\effin{\adj_j} A_j}{t_j}{\sigs}{\venv_j})_{j}\} \\
   (\bindsc{r_{k,j}}{\effin{\adj_j} A_j}{t_j}{\sigs}{\venv_j})_{j}}
   %% \forall i < k.\exists j.r_{i, j} \# t_j}
   %% l \text{~is minimal}}
  {\cu (\thunk{((r_{i,j})_j \to n_i)_i} : \thunk{\many{\effin{\adj} A
        \to}~\effout{\sigs}B})~\many{t}; c_y \redtou \cu ((\many{\venv}(n_k) :
    B)); c_y + 1}


\caption{\textsf{R-Handle} with counting.}
\label{fig:r-handle-counting}
%% \figrule
\end{figure}


\begin{figure}[h]
%% \figrule
%% \flushleft
$\boxed{m \redtou m'} $
\begin{mathpar}

\inferrule[R-Yield]
  { c_y \geq t_y }
  { m; {c_y} \redtou \key{let}~\textsf{yield} : [\textsf{Yield}]\textsf{Unit} =
    y \key{in}~m; 0}

\end{mathpar}

\caption{Inserting Yields, when over counter}
\label{fig:insert-yield-counting}
%% \figrule
\end{figure}


\section{Counting --- Take Two}


In the final semantics we count up through the number of \textsc{R-Handle} uses.
This lets us track when to next insert a yield.

We use a counter for this, labelled $\counter$ in the semantics. This counter
essentially has two states; it is either simply counting, i.e. is $\justc{n}$
for some $n$ or is a message to yield as soon as possible, i.e. $\yieldc$.

To increment this counter, we use a slightly modified version of addition,
denoted $\plusc$. This is simply defined as

\begin{equations}
  x \plusc y =
          \left\{ \ba{@{}l@{\quad}l@{}}
              \justc{x + y} & \text{if } x + y \leq \threshc \\
              \yieldc & \text{otherwise}
          \right.
\end{equations}

\noindent where $\threshc$ is the threshold at which we force a yield. Thus the
updated semantics for \textsc{R-Handle} follows.

\todo{Talk about how everything else implicitly just passes by; only if the
  counter is in the form $\justc{n}$.}

\textsc{R-Handle-Count} in Figure~\ref{fig:r-handle-counting-2} expresses that
when handling, if our counter is of the form $\justc{n}$ --- i.e. we do not have
to yield --- we perform the handling step as usual and increment the counter,
potentiall yielding if we need to. \textsc{R-Yield-Can} in
Figure~\ref{fig:insert-yield-counter} expresses that if the counter is in the
form $\yieldc$ and the evaluation context allows us to yield then we \emph{must}
yield, and reset the counter to 0. \textsc{R-Yield-Can't} says that if the
evaluation context does not permit yielding, but $m$ would otherwise reduce to
some term $m'$, then we perform that reduction inside the context. This is
important; we do not block the rest of the code from reducing if we need to
yield. The programmer may of course never want to yield; they may want their
system to remain fully synchronous. As such they can simply never write the
\textsf{Yield} interface in their program anywhere and the program will not
yield.

The rest of the operational semantics remains the same. We adopt the syntactic
sugar that any non-labelled transition $m \redtou m'$ (resp. $\redtoc$) is
shorthand for $m; \justc{n} \redtou m'; \justc{n}$; that is, it only applies
when the counter is plain and non-blocking, and it leaves the counter
unmodified.

\begin{figure}
%% \figrule
\flushleft
$\boxed{m \redtou m'} \quad \boxed{n \redtoc n'} \quad \boxed{m
    \stepstou m'} \quad \boxed{n \stepstoc n'}$
\begin{mathpar}
\inferrule[R-Handle-Count]
  {% The type system should enforce this!
   %(\handles{\adj_j}{t_j})_j \\
   k = \min_i\,\{i \mid \exists \many{\venv}.(\bindsc{r_{i,j}}{\effin{\adj_j} A_j}{t_j}{\sigs}{\venv_j})_{j}\} \\
   (\bindsc{r_{k,j}}{\effin{\adj_j} A_j}{t_j}{\sigs}{\venv_j})_{j}}
   %% \forall i < k.\exists j.r_{i, j} \# t_j}
   %% l \text{~is minimal}}
  {\cu (\thunk{((r_{i,j})_j \to n_i)_i} : \thunk{\many{\effin{\adj} A
        \to}~\effout{\sigs}B})~\many{t}; \highlight{\justc{n}}
    \redtou
    \cu ((\many{\venv}(n_k) : B)); \highlight{n \plusc 1}}

\\

\caption{\textsf{R-Handle} with better counting.}
\label{fig:r-handle-counting-2}
%% \figrule
\end{figure}

\begin{figure}[h]
%% \figrule
%% \flushleft
$\boxed{m \redtou m'} $
\begin{mathpar}
\inferrule[R-Yield-Can]
          { \EF[\EC]~\allows~\textsf{Yield} }
          { \EF[\EC[m]]; \yieldc \redtou \EF[\EC[\textsf{yield!}; m]]; \justc{0} }
\\
\inferrule[R-Yield-Can't]
          { \neg (\EF[\EC]~\allows~\textsf{Yield}) \\
            m; \justc{n} \redtou m'; c' }
          { \EF[\EC[m]]; \yieldc \redtou \EF[\EC[m]]; \yieldc }
\\

\end{mathpar}

\caption{Inserting Yields when forced to.
  }
\label{fig:insert-yield-counter}
%% \figrule
\end{figure}

\section{Soundness}


\begin{figure}[h]
%% \figrule
%% \flushleft
$\boxed{m \redtou m'} $
\begin{mathpar}
\inferrule[R-Yield-Can]
          { \EC~\allows~\textsf{Yield} }
          { \EC[n]; \yieldc \redtou \EC[\textsf{yield!}; n]; \justc{0} }
\\
\inferrule[R-Yield-Can't]
          { \neg (\EC~\allows~\textsf{Yield}) \\
            n; \justc{k} \redtou n'; c' }
          { \EC[n]; \yieldc \redtou \EC[n']; \yieldc }
\\

\end{mathpar}

\caption{Inserting Yields when forced to; with newer version }
\label{fig:insert-yield-counter}
%% \figrule
\end{figure}

We now state the soundness property for our extended system, as well as the
subject reduction theorem needed for this proof. Our system is nothing more than
the system of \cite{convent2020doo} with extra rules; as such we omit most of
the details.


\begin{theorem}[Subject Reduction]\label{thm:sub-red}
\vskip
\begin{itemize}\\
\item If $\inferskgs{m}{A}$ and $m; \counter \redtou m'; \counter'$ then $\inferskgs{m'}{A}$.
\item If $\checkskgs{A}{n}$ and $n; \counter \redtoc n'; \counter'$ then $\checkskgs{A}{n'}$.
\end{itemize}
  \end{theorem}

\begin{proof}
By induction on the transitions $\redtou, \redtoc$.

We first consider the two possible states for $\counter$. If it is in the form
$\justc{n}$, then the reduction rules are simply the same as
in~\cite{convent2020doo}, as we do not change the counter. The only exception to
this is the updated \textsc{R-Handle} rule, which is essentially the same except
for modifications to the counter; regardless of the counter, the resulting term
$m'$ still remains the same type.

Thus the only new cases are \textsc{R-Yield-Can} and \textsc{R-Yield-Can't}.

\begin{itemize}
\item[\Cse] \textsc{R-Yield-Can}
  By the assumption we have that $\EC~\allows~\textsf{yield}$. This only holds
  if the context is of the form
  \[\EC[~] = \cu (v : \thunk{\many{\effin{\adj} A
      \to}~\effout{\sigs}B})~(\many{t},[~],\many{n'})\]

  Assume that
  \[\inferskgs{\cu (v : \thunk{\many{\effin{\adj} A
      \to}~\effout{\sigs}B})~(\many{t},\EC'[n],\many{n'})}{B}\]

  Then by inversion on \textsc{T-App} we have
  $\checksk{\Gamma}{\sigs'_{|\many{t}|}}{A_{|\many{t}|}}{\EC'[n]}$. We now
  require that
  $\checksk{\Gamma}{\sigs'_{|\many{t}|}}{A_{|\many{t}|}}{\EC'[\textsf{yield};
      n]}$. This follows from the assumption $\EC~\allows~\textsf{yield}$, which
  entails that $\textsf{yield}~\in~\sigs'_{|\many{t}|}$. Thus
  $\inferskgs{\EC[\textsf{yield}; n]}{B}$.

  %% So we have that
  %% \begin{mathpar}
  %%   \EC[n] = \cu (v : \thunk{\many{\effin{\adj} A
  %%     \to}~\effout{\sigs}B})~(\many{t},\EC[n],\many{n'}) \\
  %%   \[\EC[\textsf{yield!}; n] = \cu (v : \thunk{\many{\effin{\adj} A
  %%     \to}~\effout{\sigs}B})~(\many{t},\EC[\textsf{yield!}; n],\many{n'})
  %% \end{mathpar}






\item[\Cse] \textsc{R-Yield-Can't}
  This case is more straightforward. By the assumption we have that the
  evaluation frame $\EF$ does not permit yielding, but the term inside the frame
  could otherwise reduce.

  Assume $\checkskgs{A}{\EF[n]}$, and therefore $\checkskgs{A'}{n}$. By the
  assumption and subject reduction, $\checkskgs{A'}{n'}$. Then clearly
  $\checkskgs{A}{\EF[n']}$.

\end{itemize}
\end{proof}

\begin{theorem}[Type Soundness]\label{thm:soundness}
\begin{itemize}
\\
\item If $\infers{\cdot}{\cdot}{\sigs}{m}{A}$ then either $m$ is a normal form
  such that $m$ respects $\sigs$ or there exists a unique
  $\infers{\cdot}{\cdot}{\sigs}{m'}{A}$ such that $m \stepstou m'$.
\item If $\checks{\cdot}{\cdot}{\sigs}{A}{n}$ then either $n$ is a normal form
  such that $n$ respects $\sigs$ or there exists a unique
  $\checks{\cdot}{\cdot}{\sigs}{A}{n'}$ such that $n \stepstoc n'$.
\end{itemize}
%% In particular, if $\sigs = \nowt$ then either the term is a value $w$ or the
%% term can reduce by one step.
\end{theorem}

\begin{proof}
The proof proceeds by simultaneous induction on
$\infers{\cdot}{\cdot}{\sigs}{m}{A}$ and $\checks{\cdot}{\cdot}{\sigs}{A}{n}$,
with use of Theorem~\ref{thm:sub-red}.
\end{proof}


\section{``Tree'' Yielding}


\begin{figure}[h]
%% \figrule
%% \flushleft
\begin{mathpar}
\inferrule[Add-Counter]
          { \EC[n]; \bluetext{c_1, \ldots, c_n} \redtou \EC[n']; \bluetext{c_1,
              \ldots, c_n} \\
            \EF \textsf{ is handler }}
          { \EF[\EC[n]]; \bluetext{c_1, \ldots, c_n, c_{n+1}} \redtou \EF[\EC[n']]; \bluetext{c_1, \ldots, c_n, c_{n+1}}}
\\
\inferrule[Handle-In]
  { k = \min_i\,\{i \mid \exists \many{\venv}.(\bindsc{r_{i,j}}{\effin{\adj_j} A_j}{t_j}{\sigs}{\venv_j})_{j}\} \\
    (\bindsc{r_{k,j}}{\effin{\adj_j} A_j}{t_j}{\sigs}{\venv_j})_{j} \\
    \forall~j\leq~n~.~\bluetext{c_j} \not = \yieldc
  }
  { \EC[\cu (\thunk{((r_{i,j})_j \to n_i)_i} : \thunk{\many{\effin{\adj} A
          \to}~\effout{\sigs}B})~\many{t}]; \bluetext{c_1, \ldots, c_n}
    \redtou
    \EC[\cu ((\many{\venv}(n_k) : B))]; \bluetext{\succc{c_1}, \ldots, \succc{c_n}} }
\\
\inferrule[Yield-Can]
  { \EC~\allows~\Yield \\
    \forall~j\leq~(n-1)~.~\bluetext{c_j} \not = \yieldc
  }
  { \EC[m]; \bluetext{c_1, \ldots, c_{n-1}, \yield} \redtou \EC[\yield!; m];
    \bluetext{c_1, \ldots, c_{n-1}, \justc{0}}}
  \\

\inferrule[Yield-Can't]
  { \neg (\EC~\allows~\Yield) \\
    \forall~j\leq~(n-1)~.~\bluetext{c_j} \not = \yieldc \\
    \EC[m]; \bluetext{c_1, \ldots, c_{n-1}, \justc{k}}~\redtou~\EC[m']; \bluetext{c_1,
      \ldots, c_{n-1}, c_n'}
  }
  { \EC[m]; \bluetext{c_1, \ldots, c_{n-1}, \yieldc} \redtou \EC[m'];
    \bluetext{c_1, \ldots, c_{n-1}, \yieldc} }
\\
\end{mathpar}
\caption{New counting}
\end{figure}

\begin{figure}[h]
%% \figrule
%% \flushleft
\begin{mathpar}
\inferrule[Add-Counter]
          { \EC[n]; \bluetext{c_n, \ldots, c_1} \redtou \EC[n']; \bluetext{c_n,
              \ldots, c_1} \\
            \EF \textsf{ is handler }}
          { \EF[\EC[n]]; \bluetext{c_{n+1}, c_n, \ldots, c_1} \redtou
            \EF[\EC[n']]; \bluetext{c_{n+1}, c_n, \ldots, c_1}}
\\
\inferrule[Handle-In]
  { k = \min_i\,\{i \mid \exists \many{\venv}.(\bindsc{r_{i,j}}{\effin{\adj_j} A_j}{t_j}{\sigs}{\venv_j})_{j}\} \\
    (\bindsc{r_{k,j}}{\effin{\adj_j} A_j}{t_j}{\sigs}{\venv_j})_{j} \\
    \forall~j\leq~n~.~\bluetext{c_j} \not = \yieldc
  }
  { \EC[\cu (\thunk{((r_{i,j})_j \to n_i)_i} : \thunk{\many{\effin{\adj} A
          \to}~\effout{\sigs}B})~\many{t}]; \bluetext{c_n, \ldots, c_1}
    \redtou
    \EC[\cu ((\many{\venv}(n_k) : B))]; \bluetext{\succc{c_n}, \ldots, \succc{c_1}} }
\\
\inferrule[Yield-Can]
  { \EC~\allows~\Yield \\
    \forall~j\leq~(n-1)~.~\bluetext{c_j} \not = \yieldc
  }
  { \EC[m]; \bluetext{c_n, \ldots, c_2, \yield} \redtou \EC[\yield!; m];
    \bluetext{c_n, \ldots, c_2, \justc{0}}}
  \\

\inferrule[Yield-Can't]
  { \neg (\EC~\allows~\Yield) \\
    \forall~j\leq~(n-1)~.~\bluetext{c_j} \not = \yieldc \\
    \EC[m]; \bluetext{c_n, \ldots, c_2, \justc{k}}~\redtou~\EC[m']; \bluetext{c_n,
      \ldots, c_2, c_1'}
  }
  { \EC[m]; \bluetext{c_n, \ldots, c_2, \yieldc} \redtou \EC[m'];
    \bluetext{c_n, \ldots, c_2, \yieldc} }
\\
\end{mathpar}
\caption{New counting, backwards ordered though}
\end{figure}



\todo{$c_{n + 1}$ needs to be fresh in Add-Counter}

\todo{Also need to emphasise that it's not necessarily sequential counters.}

%% \inferrule[R-Handle]
%%   {% The type system should enforce this!
%%    %(\handles{\adj_j}{t_j})_j \\
%%    k = \min_i\,\{i \mid \exists \many{\venv}.(\bindsc{r_{i,j}}{\effin{\adj_j} A_j}{t_j}{\sigs}{\venv_j})_{j}\} \\
%%    (\bindsc{r_{k,j}}{\effin{\adj_j} A_j}{t_j}{\sigs}{\venv_j})_{j}}
%%    %% \forall i < k.\exists j.r_{i, j} \# t_j}
%%    %% l \text{~is minimal}}
%%   {\cu (\thunk{((r_{i,j})_j \to n_i)_i} : \thunk{\many{\effin{\adj} A \to}~\effout{\sigs}B})~\many{t} \redtou \cu ((\many{\venv}(n_k) : B)}

%% \begin{figure}[h]
%% %% \figrule
%% %% \flushleft
%% $\boxed{m \redtou m'} $
%% \begin{mathpar}
%% \inferrule[R-Yield-Can]
%%           { \EF[\EC]~\allows~\textsf{Yield} }
%%           { \EF[\EC[m]]; \yieldc \redtou \EF[\EC[\textsf{yield!}; m]]; \justc{0} }
%% \\
%% \inferrule[R-Yield-Can't]
%%           { \neg (\EF[\EC]~\allows~\textsf{Yield}) \\
%%             m; \justc{n} \redtou m'; c' }
%%           { \EF[\EC[m]]; \yieldc \redtou \EF[\EC[m]]; \yieldc }
%% \\

%% \end{mathpar}


\chapter{Use-Only Reductions}

Here we talk about how to change the operational semantics so that only uses may
reduce.

\begin{figure}[t]
%% \figrule
\begin{syntax}
\slab{uses}                    & m   &::= & \dots \mid \freeze{\EC[c~\many{R}~\many{w}]} \\
\slab{constructions}           & n   &::= & \dots \\
\slab{use values}              & u   &::= & x \gor f~\many{R} \gor \cu (v : A) \\
\slab{non-use values}          & v   &::= & k~\many{w} \gor \thunk{e} \\
\slab{construction values}     & w   &::= & \uc u \gor v \\
\slab{normal forms}            & t   &::= & w \gor \freeze{\EC[c~\many{R}~\many{w}]} \\
\slab{evaluation frames}       & \EF &::= & [~]~\many{n}
                                      \gor  u~(\many{t},[~],\many{n})
                                      \gor  \cu([~]:A) \\
                               &     &\gor& \uc [~]
                                      \gor  k~(\many{w},[~],\many{n})
                                      \gor  c~\many{R}~(\many{w},[~],\many{n}) \\
                               &     &\gor& \key{let}~f: P = [~]~\key{in}~n
                                      \gor \effin{\adapt}~[~] \\
\slab{evaluation contexts}     & \EC &::= & [~] \gor \EF[\EC] \\
\end{syntax}
\caption{Runtime Syntax for Use-Only Reductions \todo{Remove command
    uses from constructiosn?}
\todo{Change eval ctxs (i can't remember what to change to)?}}
\label{fig:runtime-syntax-use-only}
%% \figrule
\end{figure}

\begin{figure}
%% \figrule
\flushleft
%% $\boxed{m \redtou m'} \quad \boxed{n \redtoc n'} \quad \boxed{m
%%     \stepstou m'} \quad \boxed{n \stepstoc n'}$
$\boxed{m \redtou m'} \quad \quad \boxed{m \stepstou m'} $
\begin{mathpar}
\inferrule[R-Handle]
  {% The type system should enforce this!
   %(\handles{\adj_j}{t_j})_j \\
   k = \min_i\,\{i \mid \exists \many{\venv}.(\bindsc{r_{i,j}}{\effin{\adj_j} A_j}{t_j}{\sigs}{\venv_j})_{j}\} \\
   (\bindsc{r_{k,j}}{\effin{\adj_j} A_j}{t_j}{\sigs}{\venv_j})_{j}}
   %% \forall i < k.\exists j.r_{i, j} \# t_j}
   %% l \text{~is minimal}}
  {\cu (\thunk{((r_{i,j})_j \to n_i)_i} : \thunk{\many{\effin{\adj} A \to}~\effout{\sigs}B})~\many{t} \redtou \cu ((\many{\venv}(n_k) : B)}

\inferrule[R-Ascribe-Use]
  { }
  {\cu(\uc u:A) \redtou u}

%% \inferrule[R-Ascribe-Cons]
%%   { }
%%   {\uc \cu (w : A) \redtoc w}

\highlight{
\inferrule[R-Let]
  { }
  { \cu(\key{let}~f:P = w~\key{in}~n : A)
    \redtou
    \cu(n[\cu (w : P)/f] : A)
  }
}

\highlight{
\inferrule[R-LetRec]
  {\many{e = \many{\many{r} \to n}}}
  {%\vphantom{\many{\many{\many{\many{f}}}}}
   \cu(\key{letrec}~\many{f:P = e}~\key{in}~n' : A) \redtou
   \cu(n'[\many{\cu (\thunk{\many{\many{r} \to \key{letrec}~\many{f:P =
              e}~\key{in}~n}}: P)/f}] : A)}
}

\highlight{
\inferrule[R-Adapt]
  { }
  {\cu(\effin{\adapt}~w : A ) \redtou \cu(w : A)}}

\inferrule[R-Freeze-Comm]
  { }
  {c~\many{R}~\many{w} \highlight{\redtou} \freeze{c~\many{R}~\many{w}}}\\

\inferrule[R-Freeze-Frame-Use]
  {\neg(\EF[\EC] \handles c)}
  {\EF[\freeze{\EC[c~\many{R}~\many{w}]}] \redtou \freeze{\EF[\EC[c~\many{R}~\many{w}]]}}

%% \inferrule[R-Freeze-Frame-Cons]
%%   {\neg(\EF[\EC] \handles c)}
%%   {\EF[\freeze{\EC[c~\many{R}~\many{w}]}] \redtoc \freeze{\EF[\EC[c~\many{R}~\many{w}]]}}

%% \\
\inferrule[R-Lift-UU]
  {m \redtou m'}
  {\EC[m] \stepstou \EC[m']}

%% \inferrule[R-Lift-UC]
%%   {m \redtou m'}
%%   {\EC[m] \stepstoc \EC[m']}

%% \inferrule[R-Lift-CU]
%%   {n \redtoc n'}
%%   {\EC[n] \stepstou \EC[n']}

%% \inferrule[R-Lift-CC]
%%   {n \redtoc n'}
%%   {\EC[n] \stepstoc \EC[n']}
\end{mathpar}

\caption{Operational Semantics
  \todo{Can R-Ascribe-Cons just get removed?}
  \todo{Can we just move commands to be in uses?}}
\label{fig:op-sem-use-only}
%% \figrule
\end{figure}

%% You can include appendices like this:
%%
%%%%%%%%%%%%%%%%%%%%%%%%%%%%%%%%%%%%%%%%%%%%%%%%%%%%%%%
%% APPENDIX
%%%%%%%%%%%%%%%%%%%%%%%%%%%%%%%%%%%%%%%%%%%%%%%%%%%%%%%



\bibliographystyle{plain}
\bibliography{bibliography}

\appendix

\chapter{Remaining Formalisms}


\begin{figure}%% \figrule
\flushleft
$\boxed{\adjact{\sigs}{\adj}{\sigs'}}$
\begin{mathpar}
\inferrule[A-Adj]{\adjact{\sigs}{\adapt}{\sigs'} \\
  \adjact{\sigs'}{\ext}{\sigs''}}
          {\adjact{\sigs}{\adapt\pipe\ext}{\sigs''}}
\end{mathpar}
$\boxed{\adjact{\sigs}{\ext}{\sigs'}}$
\begin{mathpar}
\inferrule[A-Ext-Id]{ }
          {\adjact{\sigs}{\id}{\sigs}}

\inferrule[A-Ext-Snoc]{\adjact{\sigs}{\ext}{\sigs'} }
          {\adjact{\sigs}{\ext, \sig~\many{R}}{\sigs', \sig~\many{R}}}
\end{mathpar}
$\boxed{\adjact{\sigs}{\adapt}{\sigs'}}$
\begin{mathpar}
\inferrule[A-Adapt-Id]{ }
          {\adjact{\sigs}{\id}{\sigs}}

\inferrule[A-Adapt-Snoc]{\adjact{\sigs}{\adapt}{\sigs'} \\
    \adpcom{\sigs'}{\sig}{S}{S'}{\sigs''}}
          {\adjact{\sigs}{\adapt, \sig(S \to S')}{\sigs''}}
\end{mathpar}
$\boxed{\adpcom{\sigs}{\sig}{S}{S'}{\sigs'}}$
\begin{mathpar}
\inferrule[A-Adapt-Com]
  {\itrbnd{\sigs}{S}{\sig}{\sigs'}{\ienv} \\
   \itrinst{\ienv}{S'}{\sig}{\ext} \\
   \adjact{\sigs'}{\ext}{\sigs''}}
  {\adpcom{\sigs}{\sig}{S}{S'}{\sigs''}}
\end{mathpar}

$\boxed{\itrbnd{\sigs}{S}{\sig}{\sigs'}{\ienv}}$
\begin{mathpar}
\inferrule[I-Pat-Id]{ }
          {\itrbnd{\sigs}{\pid}{\sig}{\sigs}{s : \sigs}}

\inferrule[I-Pat-Bind]{\itrbnd{\sigs}{S}{\sig}{\sigs'}{\ienv}}
          {\itrbnd{\sigs,\sig~\many{R}}{S~a}{\sig}{\sigs'}
            {\ienv,a:\sig~\many{R}}}

\inferrule[I-Pat-Skip]{
  \itrbnd{\sigs}{S~a}{\sig}{\sigs'}{\ienv} \\
  \sig \neq \sig'}
  {\itrbnd{\sigs,\sig'~\many{R}}{S~a}{\sig}
          {\sigs',\sig'~\many{R}}{\ienv}}
\end{mathpar}


$\boxed{\itrinst{\ienv}{S}{\sig}{\ext}}$
\begin{mathpar}
\inferrule[I-Inst-Id]{s\in\meta{dom}(\ienv)}
          {\itrinst{\ienv}{\pid}{\sig}{\id}}

\inferrule[I-Inst-Lkp]{a\in\meta{dom}(\ienv) \\
  \itrinst{\ienv}{S}{\sig}{\ext} \\
  \ienv(a)=\sig~\many{R}}
          {\itrinst{\ienv}{S~a}{\sig}{\ext,\sig~\many{R}}}
\end{mathpar}
%% \caption{Action of an Adaptor's Interface Component on an Ability}
\label{fig:interface-components}



\caption{Action of an Adjustment on an Ability and Auxiliary Judgements}
\label{fig:act-adj}
%% \figrule
\end{figure}


\begin{figure} % \figrule
\flushleft

\[
\mathcal{X} ::= A \gor C \gor T \gor G \gor Z \gor R \gor P
                  \gor \seed \gor \sigs \gor \ext \gor \adapt \gor \adj
                  \gor \Gamma \gor \exists \kenva.\Gamma \gor \ienv
\]

$\boxed{\kindchecksk{\mathcal{X}}}$
%% \boxed{\kindchecksk{C}}\boxed{\kindchecksk{T}}
%% \boxed{\kindchecksk{G}}\boxed{\kindchecksk{Z}}\boxed{\kindchecksk{R}}\boxed{\kindchecksk{P}}
%% \boxed{\kindchecksk{\seed}}\boxed{\kindchecksk{\sigs}}
%% \boxed{\kindchecksk{\ext}}\boxed{\kindchecksk{\adapt}}\boxed{\kindchecksk{\adj}}
%% \boxed{\kindchecksk{S}}\boxed{\kindchecksk{\Gamma}}\boxed{\kindchecksk{\ienv}}
%% $
\begin{mathpar}
\inferrule[WF-Val]
  { }
  {\kindchecks{\kenv, X}{X}}

\inferrule[WF-Eff]
  { }
  {\kindchecks{\kenv, [E]}{E}}

\inferrule[WF-Poly]
  {\kindchecks{\kenv, \many{Z}}{A}}
  {\kindchecks{\kenv}{\forall \many{Z}.A}}
\\
\inferrule[WF-Data]
  {(\kindchecksk{R})_i}
  {\kindchecksk{D~\many{R}}}

\inferrule[WF-Thunk]
  {\kindchecksk{C}}
  {\kindchecksk{\thunk{C}}}

\inferrule[WF-Comp]
  {(\kindchecksk{T})_i \\ \kindchecksk{G}}
  {\kindchecksk{\many{T \to}~ G}}

\inferrule[WF-Arg]
  {\kindchecksk{\adj} \\ \kindchecksk{A}}
  {\kindchecksk{\effin{\adj}A}}

\inferrule[WF-Ret]
  {\kindchecksk{\sigs} \\ \kindchecksk{A}}
  {\kindchecksk{\effout{\sigs}A}}

\inferrule[WF-Ability]
  {\kindchecksk{\sigs}}
  {\kindchecksk{[\sigs]}}

\inferrule[WF-Pure]
  { }
  {\kindchecksk{\nowt}}

\inferrule[WF-Id]
  { }
  {\kindchecksk{\id}}

\inferrule[WF-Ext]
  {\kindchecksk{\ext} \\ (\kindchecksk{R})_i}
  {\kindchecksk{\ext, \sig~\many{R}}}

\inferrule[WF-Adapt]
  {\kindchecksk{\adapt}}
  {\kindchecksk{\adapt, \sig~(S \to S')}}
\\
\inferrule[WF-Empty]
  { }
  {\kindchecksk{\cdot}}

\inferrule[WF-Mono]
  {\kindchecksk{\Gamma} \\ \kindchecksk{A}}
  {\kindchecksk{\Gamma, x : A}}

\inferrule[WF-Poly]
  {\kindchecksk{\Gamma} \\ \kindchecksk{P}}
  {\kindchecksk{\Gamma, f : P}}
\\

\inferrule[WF-Existential]
  {\kindchecks{\kenv, \kenva}{\Gamma}}
  {\kindchecksk{\exists \kenva.\Gamma}}

\inferrule[WF-Interface]
  {\kindchecksk{\ienv} \\ (\kindchecksk{R})_i}
  {\kindchecksk{\ienv, x : \sig~\many{R}}}

\end{mathpar}


\caption{Well-Formedness Rules}
\label{fig:well-formedness}

% \figrule
\end{figure}

\begin{figure} % \figrule
\flushleft
$\boxed{\matchesvk{A}{p}{\Gamma}}$
\begin{mathpar}
\inferrule[P-Var]
  { }
  {\matchesvk{A}{x}{x:A}}

\inferrule[P-Data]
  {k~\many{A} \in D~\many{R} \\
   (\matchesvk{A_i}{p_i}{\Gamma})_i}
  {\matchesvk{D~\many{R}}{k~\many{p}}{\many{\Gamma}}}
\end{mathpar}
$\boxed{\matchesck{T}{r}{\sigs}{\exists \kenva.\Gamma}}$
\begin{mathpar}
\inferrule[P-Value]
  {\adjact{\sigs}{\adj}{\sigs'} \\ \matchesvk{A}{p}{\Gamma}}
  {\matchesck{\effin{\adj}A}{p}{\sigs}{\Gamma}}

\inferrule[P-CatchAll]
  {\adjact{\sigs}{\adj}{\sigs'}}
  {\matchesck{\effin{\adj}A}{\effin{x}}{\sigs}{x:{\thunk{\effout{\sigs'}A}}}}

\inferrule[P-Command]
  {
   \adjact{\sigs}{\adj}{\sigs'} \\
   \adj = \adapt\pipe\ext \\
   c:\forall \many{Z}.\many{A \to} B \in \ext \\
   (\matchesv{\kenv, \many{Z}}{A_i}{p_i}{\Gamma_i})_i}
  {\matchesc{\kenv}
            {\effin{\adj}B'}
            {\effin{\handle{c~\many{p}}{z}}}
            {\sigs}
            {\exists \many{Z}.\many{\Gamma}, z:\{\effin{\id\pipe\id}B \to \effout{\sigs'}B'\}}}
\end{mathpar}
\caption{Pattern Matching Typing Rules}
\label{fig:pattern-typing}
%% \figrule
\end{figure}

% \section{First section}
% 
% Markers do not have to consider appendices. Make sure that your contributions
% are made clear in the main body of the dissertation (within the page limit).

\end{document}
